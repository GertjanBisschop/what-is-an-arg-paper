\documentclass{article}
\usepackage[round]{natbib}
\usepackage{amsmath,amssymb,amsfonts}%
\usepackage{geometry}%
\usepackage{color}
\usepackage{graphicx}
\usepackage{authblk}
\usepackage{nameref}
\usepackage[right]{lineno}

\begin{document}

\linenumbers
\title{What is an Ancestral Recombination Graph?}
% First authors
\author{Author McAuthory}
% Corresponding

\maketitle

% JK: this is a rough first pass for a slightly different paper. Needs
% substantial revision.
\begin{abstract}
It has recently become possible to infer genetic ancestry in the presence of
recombination at scale for the first time. This has created exciting
possibilities, with many downstream applications becoming possible based on
these inferred genealogies. The inferred genetic ancestries are usually
referred to as Ancestral Recombination Graphs, or ARGs. Although initially well
defined as a graphical representation of the coalescent with recombination
stochastic process, the interpretation has become unclear as it is now
also understood to represent a particular realisation of a genetic
ancestry.
Inference methods do not all infer the same information:
although all output marginal trees along the genome, there is a great deal of
variation in how much information is inferred and retained in the relationships
between the trees. An important programme of work over the coming years is to
develop and refine ancestry inference methods, assessing the relative strengths
and weaknesses of the various approaches. Using the blanket term ARG hides the
important differences in the approaches, and hampers our ability to compare and
improve inference methods. In this paper we discuss the different approaches we
may use to encode genetic ancestry, and of the differing amounts of information
about the ancestral process that we can represent and hope to observe. We
classify existing inference methods according to the type of ancestry they
infer, and discuss the strengths and weaknesses of the inferred structure for
downstream applications.
\end{abstract}

\textbf{Keywords:} Ancestral Recombination Graphs

\section*{Introduction}

% HYW: this is a rough first pass, aiming to be concise. We should flash out the "recombination" aspect of ARGs here a bit, I think

For a long time, the Ancestral Recombination Graph (ARG) has been an object of
fascination to those interested in theoretical evolutionary genetics, as it
captures all knowable genetic history of a collection of individual genomes.
Recently advances in our ability to infer ARGs has led to a resurgence of
interest in this field. However, these have also given rise to certain
terminological confusion, which we aim to clear up in this paper.

One major source of confusion is the use of the term "ARG" to refer to both a
backwards-in-time generative process (Griffiths) and a structure which can be
created by that process
\citep[e.g.][]{minichiello2006mapping,mathieson2020ancestry}.

\section*{Ancestral graphs and stochastic processes}

% See https://github.com/tskit-dev/what-is-an-arg-paper/discussions/13
% for discussions on this section

The coalescent
process~\citep{kingman1982coalescent,kingman1982genealogy,hudson1983testing,
tajima1983evolutionary}

The Ancestral Recombination Graph (ARG) was introduced by
Griffiths~\citep{griffiths1991two,griffiths1997ancestral}.
The ``big'' ARG~\citep{ethier1990two},
and the ``little ARG'' traversed by
Hudson's algorithm~\citep{hudson1983properties}.


In the theoretical literature, ancestral graphs such as the ARG refer to the
ancestral \emph{process} rather than the outcome of a process.
\begin{itemize}
\item Ancestral Selection
Graph~\citep{krone1997ancestral,neuhauser1999ancestral} models selection
as a graph with virtual lineages [etc etc.]. Importantly, the graph structure
of the ASG does \emph{not} correspond to a realised ancestry, but rather
models possible outcomes.
Other papers possibly worth citing:\citep{etheridge2009coalescent}
\item Ancestral Gene Transfer Graph~\citep{baumdicker2014infinitely}
\end{itemize}

\section*{The ARG as a structure}

\citep[e.g.][]{minichiello2006mapping,mathieson2020ancestry}.

% HYW: this is a rough first pass. We should take care not to assume that the tree sequence
% approach is the only encoding

The other use of the the term ARG, and the one for which we suggest the term be
reserved, is to describe the *structure* resulting from the genetic process of
inheritance. This structure, like any other graph, consists of *nodes*
connected by *edges*. In the context of genetics, the nodes represent (haploid)
genomes, and the edges represent paths of inheritance. Although for theoretical
purposes a graph like this can be created by backwards-in-time simulations
using the CwR, in reality it is created forwards-in-time, as a population
evolves.

Since genetic inheritance is unidirectional, the edges in this graph structure
are *directed*, with a "parent" and "child" node. Moreover, there is a strict
temporal order, so that children cannot be their own parents: in other words,
there are no cycles in the graph, and it is technically a form of "directed
acyclic graph" (DAG).

% HYW: NB: worded carefully below, as it should also be possible to encode ARGs even if there is not a single fixed coordinate system

There is one key feature which distinguishes an ARG for other graphs. Although
a genome (graph node) may have more than one parent, the particulate nature of
genetic inheritance means that each letter in the genome can only come from a
*single* parent. Hence if we compare the equivalent letters from multiple
extant genomes, their paths of inheritance must form a *tree* (which,
incidentally, is why phylogenetics plays such an important role in evolutionary
theory). As we shall argue, the ability to generate these "local trees" from an
ARG is of key importance.

When describing an ARG, the primary issue, from a practical point of view, is
how to annotate the graph with the details of which piece of genome has come
from which parent. There are two possibilities: the annotations can be
associated either with the nodes, or with the edges.

\subsection*{Node annotated recombination}

In the original (Griffiths) formulation, nodes which have multiple parents are
annotated with a breakpoint, representing a genomic position. The genome
lying to the left of that position is then taken to come from one parent, while
the genome lying to the right is taken to come from the other parent. This is
an elegantly concise encoding of the parental source, and is used, for example
in the ARGweaver .arg format. However, it suffers from three principal
drawbacks:


1. It assumes only a single breakpoint per node: problematic for representing
e.g. gene conversion or multiple chromosomes (but this could be circumvented by
having multiple breakpoints)
2. Nodes can only have a maximum of 2 parents. At
first sight, this seems a reasonable limitation, but it turns out that it can
be helpful to collapse multiple nodes into one, which potentially results in
more than two parents for any given node (see section XXXX)
3. Generating local
trees from such an encoding is time consuming (XXX describe *how* time
consuming - i,e. scaling properties with number of nodes - and why this is,
theoretically)

\subsection*{Edge annotated recombination}

An alternative approach to annotating nodes with the parent of origin of each
section of genome is to annotate each edge with the genomic region which is
inherited via that route. This approach was first introduced by Kelleher et al
(cite msprime paper) and now forms the basis of a comprehensive software
library.

% HYW: should discuss the link to biology & mitosis/meiosis: a "node" is
% strictly a cell - it is shorthand to associate it with an "individual"

\section*{ARG inference methods}
Using the terminology developed here to classify ARGs, we review methods developed
to infer ARGs. We focus our attention on some recent methods and discuss the
properties of the ARGs inferred.

\begin{itemize}
\item Early methods \citep{hein1990reconstructing}
\item Those sampling from the coalescent with recombination
\citep{griffiths1996ancestral,kuhner2000maximum,fearnhead2001estimating},
or SMC
\citep{rasmussen2014genome}
\item Hueristic methods~\citep{song2005efficient,minichiello2006mapping,
parida2008estimating,fallon2013acg,mirzaei2016rent}
\item Discussions of the complexity of inferring ARGs (but we need to
dig into this: is this the complexity of sampling from the coalescent,
finding a ``min ARG'', or something else?
\citep{bordewich2005computational,wang2001perfect,hein1990reconstructing,song2005constructing}
\item Recent methods: Relate, tsinfer, tsdate, etc.
\end{itemize}

\bibliographystyle{plainnat}
\bibliography{paper}

\end{document}
